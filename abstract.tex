\chapter*{Abstract}

Health inequalities persist despite decades of effort to reduce them.
Faced with a reduction in public spending, contraction of the welfare state, and rising inequality it is likely that health inequalities will increase for years to come.
A better understanding of health resilience, which areas and individuals are resilient, and what factors might `protect' their health outcomes might help develop policies to break down the link between disadvantage and health.

This research contributes to the understanding of health resilience in the case study area of Doncaster, South Yorkshire.
As a former mining town, Doncaster is exposed to significant economic disadvantage reflected in many settlements across the North East, North West, Midlands, and South Wales.
Previous geographical research into health resilience has been limited either to small--area information with basic health outcomes, or more sophisticated measures of health outcomes but geographically aggregated to large regions.
Using spatial microsimulation, I present the first estimate of health resilience at the small--area level using measures of health previously inaccessible to researchers.

This is complemented by a systematic scoping literature review of measures hypothesised to affect health resilience.
I simulate a broad range of these alongside clinical depression and income to explore a more comprehensive range of factors than have previously been possible.
This includes small--area and individual--level factors, which are difficult to separate.

I conclude by comparing geographical proximity of a number of health amenities to resilient and non--resilient areas in Doncaster, and by evaluating local and national policies such as Universal Credit and their likely effect on the residents of Doncaster and their resilience.
